\documentclass{article}

\usepackage{mathrsfs,amsmath}
\usepackage{xcolor}
\usepackage{titlesec}
\usepackage{listings}
\usepackage{syntax}
\usepackage{pythonhighlighting}

\usepackage{graphicx}

\graphicspath{ {./assets/} }

\usepackage[margin=1.4in]{geometry}

\title{Handout \#3 | CS 471} 
\author{Jared Dyreson\\ 
        California State University, Fullerton}

\DeclareRobustCommand{\bowtie}{%
  \mathrel\triangleright\joinrel\mathrel\triangleleft}


\usepackage [english]{babel}
\usepackage [autostyle, english = american]{csquotes}
\MakeOuterQuote{"}

\titlespacing*{\section}
{0pt}{5.5ex plus 1ex minus .2ex}{4.3ex plus .2ex}
\titlespacing*{\subsection}
{0pt}{5.5ex plus 1ex minus .2ex}{4.3ex plus .2ex}

\usepackage{hyperref}
\hypersetup{
    colorlinks,
    citecolor=black,
    filecolor=black,
    linkcolor=black,
    urlcolor=black
}

\begin{document}

\maketitle
\tableofcontents

\newpage

\section{Questions}

\begin{enumerate}

\item What are nodal, queueing, transmission, processing, and propagation delays? 
\begin{itemize}
\item Nodal ($d_{proc}$): the time that a node spends processing a packet
\item Queueing: the sum of the delays encountered by a packet between the time of insertion into the network and the time of delivery to the destination.
\item Transmission: the amount of time required to push all the packet's bits through a wire
\item Processing: the time it takes routers to process the packet header.
\item Propagation: measure of the time required for a signal to propagate from one end of the circuit to the other.
\end{itemize}

\item What is the difference between transmission and propagation delays?
\begin{itemize}
\item Transmission delays refer to the entire packet being delivered when the propagation delay is in reference to an individual bit sent through the circuit.
\end{itemize}

\item Two systems are connected by a router. Both systems and the router have transmission rates of 1,000 BPS. Each link has a propagation delay of 10ms. Also, it takes the router 2ms to process the packet. Suppose the first system want to send a 10,000 bit packet to the second system. How long will it take before the receiver system obtain the entire packet?
\begin{itemize}
\item Regardless of delays, it will take at least 10 seconds (10000 milliseconds) to transmit each way. Each way will also have a delay of 10 milliseconds along with an additional 2 milliseconds for processing. Therefore it will take:
$$ 20+ 0.2 + 0.002 = 20.022 $$ seconds for the packet to be sent in this ecosystem.
\end{itemize}


\item Consider two hosts, $A$ and $B$, connected by a single link of rate $R$ bps. Suppose that the two hosts are separated by $m$ meters, and suppose the propagation speed along the links is $s$ meters/second. Host $A$ is to send a packet of size $L$ bits to host $B$.
\begin{itemize}
\item Express the propagation delay, $d_{prop}$, in terms of $m$ and $s$.

\begin{itemize}
\item $d_{prop} = \frac{m}{s}$
\end{itemize}

\item Determine the transmission time of the packet, $d_{trans}$ in terms of $L$ and $R$
\begin{itemize}
\item $d_{trans} = \frac{L \text{ (bits) }}{R \text{ (bits/second) }}$
\end{itemize}

\item Ignoring processing and queueing delays, obtain an expression for the end-to-end delay
\begin{itemize}
\item Originally, $d_{nodal}$ can be expressed using the following: $d_{nodal} = d_{proc} + d_{queue} + d_{trans} + d_{prop}$
\item Therefore, in this instance $d_{nodal}$ can be written as: $d_{nodal} = d_{trans} + d_{prop} \implies d_{nodal} = L + R$
\end{itemize}

\newpage

\item Suppose host A begins to transmit the packet at $t = 0$. At $t = d_{trans}$, where is the last bit of the packet?
\begin{itemize}
\item The last bit of the packet has already been sent to the destination node
\end{itemize}

\item Suppose $d_{prop} > d_{trans}$. At $t = d_{trans}$, where is the first bit of the packet?
\begin{itemize}
\item Host A has already finished transmitting the last bit of the packet but since the initial condition says the destination has not yet received the first bit.
\end{itemize}

\item Suppose $d_{prop} < d_{trans}$. At $t = d_{trans}$, where is the first bit of the packet?
\begin{itemize}
\item The first bit of the packet has reached the destination node.  
\end{itemize}

\item Suppose $s = 2.5 \times 10^{8}$, $L = 120 \text{ bits }$, and $R = 56 \text{ kbps } \implies 56000 \text{ bps }$. Find the distance $m$ so that $d_{prop} = d_{trans}:$
$$ \frac{L}{R} = \frac{m}{s}$$
$$ \frac{120}{56000} = \frac{m}{2.5 \times 10^{8}}$$
$$m = (\frac{120}{56000} \times (2.5 \times 10^{8})) \div 1000$$
$$m \cong 535.714 \text{ km }$$
\end{itemize}


\end{enumerate}

\end{document}


